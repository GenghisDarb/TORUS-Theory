% Options for packages loaded elsewhere
\PassOptionsToPackage{unicode}{hyperref}
\PassOptionsToPackage{hyphens}{url}
%
\documentclass[
]{article}
\usepackage{amsmath,amssymb}
\usepackage{iftex}
\ifPDFTeX
  \usepackage[T1]{fontenc}
  \usepackage[utf8]{inputenc}
  \usepackage{textcomp} % provide euro and other symbols
\else % if luatex or xetex
  \usepackage{unicode-math} % this also loads fontspec
  \defaultfontfeatures{Scale=MatchLowercase}
  \defaultfontfeatures[\rmfamily]{Ligatures=TeX,Scale=1}
\fi
\usepackage{lmodern}
\ifPDFTeX\else
  % xetex/luatex font selection
\fi
% Use upquote if available, for straight quotes in verbatim environments
\IfFileExists{upquote.sty}{\usepackage{upquote}}{}
\IfFileExists{microtype.sty}{% use microtype if available
  \usepackage[]{microtype}
  \UseMicrotypeSet[protrusion]{basicmath} % disable protrusion for tt fonts
}{}
\makeatletter
\@ifundefined{KOMAClassName}{% if non-KOMA class
  \IfFileExists{parskip.sty}{%
    \usepackage{parskip}
  }{% else
    \setlength{\parindent}{0pt}
    \setlength{\parskip}{6pt plus 2pt minus 1pt}}
}{% if KOMA class
  \KOMAoptions{parskip=half}}
\makeatother
\usepackage{xcolor}
\usepackage{longtable,booktabs,array}
\usepackage{calc} % for calculating minipage widths
% Correct order of tables after \paragraph or \subparagraph
\usepackage{etoolbox}
\makeatletter
\patchcmd\longtable{\par}{\if@noskipsec\mbox{}\fi\par}{}{}
\makeatother
% Allow footnotes in longtable head/foot
\IfFileExists{footnotehyper.sty}{\usepackage{footnotehyper}}{\usepackage{footnote}}
\makesavenoteenv{longtable}
\setlength{\emergencystretch}{3em} % prevent overfull lines
\providecommand{\tightlist}{%
  \setlength{\itemsep}{0pt}\setlength{\parskip}{0pt}}
\setcounter{secnumdepth}{-\maxdimen} % remove section numbering
\ifLuaTeX
  \usepackage{selnolig}  % disable illegal ligatures
\fi
\IfFileExists{bookmark.sty}{\usepackage{bookmark}}{\usepackage{hyperref}}
\IfFileExists{xurl.sty}{\usepackage{xurl}}{} % add URL line breaks if available
\urlstyle{same}
\hypersetup{
  hidelinks,
  pdfcreator={LaTeX via pandoc}}

\author{}
\date{}

input{macros/torus_macros.tex}
%% Auto-patch: missing macros & safer compile
\ProvidesFile{torus_book_preamble_patch}[2025/06/02 TORUS ad-hoc fixes]

% ---- 1. macros that were undefined ----------------------
\newcommand{\LambdaCDM}{\ensuremath{\Lambda\text{CDM}}}
\newcommand{\LCDM}{\LambdaCDM} % alias if used elsewhere

% ---- 2. show deeper error context -----------------------
\errorcontextlines=100

% ---- 3. ad-hoc fixes for DOCX conversion artifacts ----
\newcommand{\hbarc}{\hbar c}
\newcommand{\textless}{<}
\newcommand{\textgreater}{>}
\newcommand{\textless/sub}{\ensuremath{_{<}}}
\newcommand{\textgreater/sub}{\ensuremath{_{>}}}

% ---- 4. more ad-hoc fixes for undefined macros ----
\newcommand{\Lambdarec}{\Lambda_{\mathrm{rec}}}
\newcommand{\real}[1]{\mathrm{Re}\left(#1\right)}

% ---- 5. Unicode and font support for XeLaTeX ----
\usepackage{fontspec}
\usepackage{unicode-math}
\setmainfont{Latin Modern Roman}
\setmathfont{Latin Modern Math}

% ---- 6. Additional robust error surfacing ----
% Show all undefined references and citations as errors
\AtEndDocument{%
  \if@filesw\immediate\write\@mainaux{\string\@input{\jobname.aux}}\fi
  \ifx\@undefined\undefined\errmessage{Undefined macro found!}\fi
}

% Optionally, force fatal error on undefined control sequence (for CI)
% \makeatletter
% \def\@undefined#1{\errmessage{Undefined control sequence: #1}}
% \makeatother

% ---- 7. Add any further missing macros or fixes below ----

\begin{document}

\textbf{Appendix B: TORUS 14-Dimensional Hierarchy and Fundamental
Constants}

This appendix presents a reference hierarchy for TORUS Theory's 14
dimensions (0D through 13D). Table \textbf{B-1} below summarizes each
dimensional level, the fundamental constant associated with that stage
(with its symbol and approximate value or notation), and a brief
description of its physical meaning and role as an ``anchor'' in the
recursion cycle. These constants range from the extremely small quantum
scales (e.g. Planck time and length) up to the cosmic scale (observable
universe size and age), and they are \textbf{not arbitrary} -- each
constant is related to others through mathematical relationships, and
the highest-level constants feed back into the lowest level to complete
the toroidal recursion\hspace{0pt}.

\textbf{Table B-1. TORUS Dimensional Hierarchy (0D--13D) with Key
Constants}\hspace{0pt}

\begin{longtable}[]{@{}@{}}
\toprule\noalign{}
\endhead
\bottomrule\noalign{}
\endlastfoot
 \\
 \\
 \\
 \\
 \\
 \\
 \\
 \\
 \\
 \\
 \\
 \\
 \\
 \\
\end{longtable}

Each of the above constants defines a \textbf{new layer of physical
reality} in the TORUS framework. Starting from 0D's tiny dimensionless
coupling, the hierarchy builds upward through familiar fundamental units
(time, length, mass, etc.) and then into thermodynamic and cosmic
scales. Crucially, these constants are interrelated across dimensions:
lower-dimensional constants combine to give rise to higher-dimensional
ones, and the highest levels feed back into the lowest, ensuring the
\textbf{closure of the toroidal recursion} (after 13D, the ``next'' step
loops back to 0D rather than introducing an independent
14D)\hspace{0pt}. Below, we elaborate on each dimension's constant with
its physical interpretation, derivation context, and how it harmonizes
with other constants across the 14D scale.

\textbf{0D -- Origin Coupling Constant (Seed Dimensionless Parameter)}

\textbf{Constant \& Value:} A fundamental \textbf{dimensionless
coupling} of order \textasciitilde0.0073 (approximately
1/137)\hspace{0pt}. In magnitude, this is essentially the same as the
electromagnetic fine-structure constant \alpha \approx
1/137.03599\ldots\hspace{0pt}. TORUS adopts this constant at 0D as an
\emph{analog} of the fine-structure constant -- it represents the
initial ``seed'' interaction strength at the origin of the recursion
cycle.

\textbf{Physical Meaning:} At 0D (zero dimensions), we have an
\textbf{origin point} with no extent in space or time. This tiny
coupling is the only defining parameter of that stage, and it
\textbf{``seeds'' the entire cycle} with a baseline interaction
strength\hspace{0pt}. In other words, even in a 0-dimensional state
there is a nonzero propensity for physical interaction -- a primordial
kernel from which higher-dimensional structures will grow. The smallness
of this constant (\textasciitilde10\^{}-2) means the cycle starts
gently: the initial coupling is weak, providing a delicate starting
point that will amplify through subsequent dimensions\hspace{0pt}.

\textbf{Anchor Role in Recursion:} Being dimensionless and at the start,
the 0D constant anchors the \textbf{micro end} of the TORUS loop. Many
of the higher-dimensional constants relate back to this seed value
through mathematical ratios or as part of larger dimensionless
combinations. Notably, TORUS postulates that the \textbf{final 13D
constant (cosmic time)} will inversely mirror the 0D
constant\hspace{0pt}. In essence, the extremely small coupling at 0D
finds its complement in an extremely large time/length at 13D, helping
to close the recursion loop. This idea is that if one ``runs'' the tiny
coupling through all the transformations of the 14-stage cycle, by the
end (13D) the product of factors yields a dimensionless unity, which
then effectively resets the next cycle\hspace{0pt}. The interplay
between 0D and 13D is thus a cornerstone of TORUS's \textbf{toroidal
closure}: the output of the highest dimension feeds back as the input to
the lowest, ensuring consistency. In summary, 0D contributes a small but
crucial dimensionless number that sets the stage for the universe's
parameters, and after the full recursion up to 13D, the universe
``closes the loop'' by using the 13D result to regenerate a 0D-like
state for a new cycle\hspace{0pt}.

\textbf{1D -- Temporal Quantum (Fundamental Time Interval)}

\textbf{Constant \& Value:} The \textbf{Planck time} \emph{t}ₚ,
approximately 5.39 × 10\^{}-44 seconds\hspace{0pt}. This is the smallest
meaningful unit of time in known physics, effectively the ``quantum'' of
time. TORUS designates \emph{t}ₚ as the fundamental time interval at the
1D level.

\textbf{Physical Meaning:} At 1D, one degree of freedom is introduced --
\textbf{time}. The 1D constant represents the minimal ``tick'' of time,
i.e. the shortest duration that makes physical sense in the
model\hspace{0pt}. Below this scale, the concept of a smooth time
continuum breaks down; 1D provides a discrete stepping for the
recursion. We can think of \emph{t}ₚ as the \textbf{frame rate of the
universe's progression}\hspace{0pt}: each step of the TORUS recursion
advances by one Planck-time increment. This means all higher processes
count time in units of this fundamental interval.

\textbf{Harmonization Across Scales:} The Planck time is intimately
linked with other constants to ensure consistency. A key relation is
with the speed of light (4D constant \emph{c}): one Planck time
multiplied by \emph{c} yields one Planck length (2D constant): \emph{c}
× \emph{t}ₚ \approx \emph{\ell}ₚ\hspace{0pt}. This built-in linkage means that in
one fundamental time tick, light travels one fundamental length. It is a
direct embedding of Einstein's \textbf{space--time relation} at the
smallest scale. The 1D constant also sets a base frequency scale -- its
inverse (\approx 1.854×10\^{}43 s\^{}-1) is the ``Planck frequency.'' Using
this frequency with the 5D constant (Planck's \emph{h}) reproduces the
Planck energy: \emph{h} × (1/\emph{t}ₚ) \textasciitilde{} 1.23×10\^{}10
J, on the order of \emph{m}ₚ \emph{c}\^{}2\hspace{0pt}. Thus, one
oscillation per \emph{t}ₚ carries roughly one Planck mass-energy,
showing how 1D (time) combines with 5D (action) to connect to 3D
(mass-energy). Furthermore, the enormous cosmic time (13D
\emph{T}\textless sub\textgreater U\textless/sub\textgreater) is
essentially a colossal multiple of this 1D tick. In fact,
\emph{T}\textless sub\textgreater U\textless/sub\textgreater/\emph{t}ₚ
\textasciitilde{} 8×10\^{}60, a huge dimensionless number that
intriguingly can be factored into products of other fundamental ratios
(as discussed at 13D)\hspace{0pt}. All these connections underscore that
\emph{t}ₚ is not an isolated parameter; it sits at the foundation of a
hierarchy where \textbf{time scales from 10\^{}-44 s to 10\^{}17 s are
related} by the structure of the recursion.

\textbf{2D -- Spatial Quantum (Fundamental Length Scale)}

\textbf{Constant \& Value:} The \textbf{Planck length} \emph{\ell}ₚ, about
1.616 × 10\^{}-35 meters\hspace{0pt}. This is the smallest meaningful
unit of length, effectively the ``quantum'' of space in the model.

\textbf{Physical Meaning:} At 2D, the recursion adds \textbf{spatial
extent}. The 2D constant \emph{\ell}ₚ defines the minimal length scale --
roughly the size of a ``pixel'' of space. No structure can be smaller
than this length in TORUS; it represents the granularity of spacetime
(below \emph{\ell}ₚ, classical geometry ceases to make sense, due to
quantum gravitational fuzziness). With 1D time in place, introducing a
fundamental length means we now have a basis for a space-time framework
at the tiniest scale. In effect, \emph{\ell}ₚ is the length at which space
itself is quantized, aligning with the notion that around 10\^{}-35 m,
quantum foam and space-time discreteness become important.

\textbf{Derivation \& Relations:} The Planck length is not chosen
arbitrarily but emerges from the interplay of more basic constants. As
mentioned, it is linked to the Planck time by \emph{\ell}ₚ = \emph{c} ·
\emph{t}ₚ, ensuring that space and time units are consistent (one Planck
time of light travel equals one Planck length). Moreover, \emph{\ell}ₚ sits
at the crossroads of quantum mechanics and gravity: it is approximately
the scale at which a particle's \textbf{Compton wavelength} (quantum
uncertainty in position) equals its \textbf{Schwarzschild radius}
(gravitational radius). This happens for a particle of Planck mass (3D
constant), illustrating that when you plug in \emph{m}ₚ, the
characteristic quantum length \hbar/(mₚ c) and gravitational length 2G
mₚ/c\^{}2 both come out to \textasciitilde1.6×10\^{}-35 m\hspace{0pt}.
That duality is essentially the definition of the Planck length in terms
of \hbar, G, and c, and TORUS encapsulates it as the point where the 2D, 3D,
and 9D constants intersect. Thus, 2D's constant ties together the
presence of time (1D) and light speed (4D) with quantum (\hbar at 5D) and
gravity (G at 9D) in a single fundamental scale\hspace{0pt}. As the
recursion proceeds to larger scales, \emph{\ell}ₚ acts as the \textbf{base
unit}: all macroscopic lengths (atomic scales, meter scales, etc.) are
multiples of this fundamental quantum of space. Ultimately, the
observable universe's size (12D) is an enormous multiple of \emph{\ell}ₚ,
and TORUS emphasizes that the product of the smallest and largest
lengths is not random but yields a meaningful dimensionless number (see
12D)\hspace{0pt}.

\textbf{3D -- Mass--Energy Unit (Quantum--Gravity Crossover Scale)}

\textbf{Constant \& Value:} The \textbf{Planck mass} \emph{m}ₚ, roughly
2.176 × 10\^{}-8 kilograms\hspace{0pt}, equivalent to about 2.0 ×
10\^{}9 Joules of energy (\emph{m}ₚ c\^{}2). This is the fundamental
mass-energy unit in the TORUS recursion.

\textbf{Physical Meaning:} By 3D, having time (1D) and length (2D) in
place, the recursion introduces \textbf{mass and energy}. The 3D
constant \emph{m}ₚ represents a pivotal scale where quantum effects and
gravitational effects are equally important. It is essentially the mass
at which an object's own gravity is as significant as its quantum
(wave-particle) nature\hspace{0pt}. Below this mass, particles are
typically in the quantum regime with negligible self-gravity; at around
this mass and above, gravitational interactions become non-negligible
even at the quantum scale. In TORUS, \emph{m}ₚ thus marks the
\textbf{threshold between the microcosm and the macrocosm}\hspace{0pt}:
it's the scale at which a particle can gravitate like a black hole and
at the same time have a quantum wavelength on the order of the Planck
length. In practical terms, this is around 21.8 micrograms --
surprisingly large for a ``fundamental'' mass (about the mass of a dust
mite or a flea's egg), yet incredibly tiny on astronomical
scales\hspace{0pt}. No known elementary particle approaches this mass;
it's a theoretical construct signaling where our conventional physics
might need unification.

\textbf{Derivation \& Cross-Links:} The Planck mass is determined by
lower-level constants together with gravity (9D). In fact, by setting
the Compton wavelength equal to the Schwarzschild radius as noted above,
one can solve for \emph{m} that satisfies \hbar/(m c) = 2Gm/c\^{}2 =
\emph{\ell}ₚ, which yields m = \emph{m}ₚ\hspace{0pt}. Another way to see
its significance is through a dimensionless combination: Gm ⁣p2/(ℏc)\approx1G
m\_\{\textbackslash!p\}\^{}2/(\textbackslash hbar c) \approx
1Gmp2\hspace{0pt}/(ℏc)\approx1\hspace{0pt}, meaning the gravitational
interaction energy of two Planck masses at Planck-length separation is
comparable to the energy of a single quantum (\hbar) times c. TORUS builds
this unity in by design: by the time we reach 3D in the hierarchy, the
constants introduced (including G from 9D and \hbar from 5D) ensure that
this combination is \textasciitilde1\hspace{0pt}. Thus \emph{m}ₚ is not
a free parameter but one fixed by earlier constants \hbar, G, and c (indeed
m ⁣p=ℏc/Gm\_\{\textbackslash!p\} =
\textbackslash sqrt\{\textbackslash hbar
c/G\}mp\hspace{0pt}=ℏc/G\hspace{0pt}). In the recursion context, the 3D
scale is supported by 2D and 4D (space and relativistic unit c, via E =
m c\^{}2) and also anticipates 9D (gravity) by defining where gravity
``turns on.'' If one accumulates enough 1D time quanta and 2D spatial
quanta worth of energy, reaching one 3D quantum of energy (∼2×10\^{}9 J)
means \textbf{self-gravity becomes noticeable}\hspace{0pt}. In summary,
3D's Planck mass ties together the foundational constants from lower
dimensions into a mass scale that bridges quantum mechanics and
gravitation, ensuring the hierarchy smoothly transitions from
quantum-dominated physics to gravity-influenced physics at this point.

\textbf{4D -- Space--Time Link (Invariant Speed of Light)}

\textbf{Constant \& Value:} The \textbf{speed of light} \emph{c},
exactly 299,792,458 m/s in vacuum (defined value)\hspace{0pt}. TORUS
takes \emph{c} as the defining constant of the 4D level.

\textbf{Physical Meaning:} At 4D, the concept of \textbf{space-time
unification} enters. While time and space were introduced at 1D and 2D,
it is the 4D constant \emph{c} that truly binds them into a single
framework. The speed of light is the conversion factor between units of
time and units of space\hspace{0pt}, effectively defining how many
meters ``correspond'' to a second. In TORUS, reaching 4D corresponds to
achieving a (3+1)-dimensional space-time with \emph{c} dictating the
structure of relativity. The presence of \emph{c} ensures that
\textbf{causality} is built into the recursion: no signal or influence
can propagate faster than this speed, at any subsequent
level\hspace{0pt}. In essence, 4D marks the stage where the universe's
fabric has a finite light-speed limit, establishing the relativistic
arena for all higher-dimensional physics to play out.

\textbf{Interrelations:} The introduction of \emph{c} solidifies links
that were already implicit. We've noted \emph{c} ties the 1D and 2D
constants by \emph{c} · \emph{t}ₚ = \emph{\ell}ₚ\hspace{0pt}, cementing the
harmony between fundamental time and length. \emph{c} also appears in
relations involving other constants: for the 3D mass-energy, \emph{c}
converts mass to energy (E = m c\^{}2), and for the 5D action quantum,
\emph{c} relates energy and wavelength (E = h c/λ)\hspace{0pt}. By
explicitly including \emph{c}, TORUS ensures that \textbf{Lorentz
invariance} (the principle of relativity) is ingrained in the theory
from 4D onward. This means all processes from here up respect the fact
that space and time coordinates mix under high-speed motion and that
\emph{c} is the same in all reference frames. Adjacently, the value of
the Planck mass (3D) and Planck time (1D) were defined using \emph{c},
and upcoming constants will frequently incorporate \emph{c} (e.g. Planck
temperature uses c in mₚ c\^{}2). By 4D, the recursion has constructed a
full space-time backdrop; any phenomena introduced at 5D and above will
occur \textbf{within this relativistic space-time}\hspace{0pt}. In
summary, \emph{c} is the \textbf{glue of spacetime} in TORUS: it links
space with time and ensures that the hierarchy conforms to the same
light-speed limit observed in reality, underpinning cause and effect at
all scales.

\textbf{5D -- Quantum of Action (Planck's Constant, \hbar)}

\textbf{Constant \& Value:} \textbf{Planck's constant} \emph{h}, which
is 6.62607015 × 10\^{}-34 J·s (exact, by SI definition)\hspace{0pt}.
Often one uses the reduced Planck constant \hbar = h/2π, but TORUS treats
\emph{h} itself as the 5D constant for simplicity. This constant
represents the smallest unit of action in quantum mechanics.

\textbf{Physical Meaning:} By the time we reach 5D, the recursion
explicitly incorporates \textbf{quantum mechanics}. Planck's constant
introduces the rule that action (energy × time, or momentum × distance)
comes in discrete packets. In other words, 5D is the stage where
nature's processes become quantized\hspace{0pt}. Before this, one could
imagine time, length, and even energy as continuous (though bounded by
Planck scales); with 5D, we recognize that not every value is allowed --
energy, angular momentum, etc., increase in jumps of size h (or related
quanta like \hbar). This adds a new degree of freedom often described as the
phase or quantum state. Essentially, 5D anchors the entire
\textbf{quantum realm}: phenomena like superposition, uncertainty, and
wave-particle duality enter, governed by this constant unit of action.

\textbf{Context and Integration:} Planck's constant ties together
earlier constants by relating energy and frequency: E = h ν. If we take
ν = 1/\emph{t}ₚ (the fundamental frequency of the 1D tick), then E = h/
\emph{t}ₚ \approx 1.23×10\^{}10 J\hspace{0pt}. Remarkably, this is on the same
order as \emph{m}ₚ c\^{}2 (\textasciitilde2×10\^{}9 J)\hspace{0pt}. Thus
one quantum oscillation at the Planck frequency carries roughly a Planck
mass-energy. This near-equality demonstrates a \textbf{harmonic
consistency}: the 5D constant and the 1D time quantum are chosen such
that h/ \emph{t}ₚ \approx \emph{m}ₚ c\^{}2\hspace{0pt}. In other words, the
fundamental energy associated with the smallest time interval aligns
with the fundamental mass-energy introduced at 3D -- showing that the
microphysical constants (\hbar, tₚ, c) work together rather than in
isolation. Planck's constant also works with the next constant, k\_B
(6D), to connect quantum and thermal physics. For example, setting a
quantum's energy h ν equal to thermal energy k\_B T leads to a
characteristic temperature; using ν = 1/tₚ yields T on the order of
10\^{}32 K, essentially the Planck temperature (10D)\hspace{0pt}.
Additionally, h and k\_B appear together in formulas like Planck's law
of blackbody radiation and the Boltzmann factor e\^{}(--E/k\_B T),
indicating 5D and 6D jointly govern quantum statistical behavior. By
sitting at 5D, Planck's constant is flanked by c (4D) which provides the
link between frequency and wavelength (as in E = h c/λ) and k\_B (6D)
which will convert energies to temperature\hspace{0pt}. This central
position means 5D connects the \textbf{microscopic oscillations} of
fields/particles to both the spacetime structure beneath (via 4D) and
the macroscopic ensembles above (via 6D). In summary, TORUS includes
\emph{h} as a fundamental step to ensure that \textbf{quantization} is a
built-in feature of the universe once spacetime is established,
seamlessly integrating classical scales with quantum rules.

\textbf{6D -- Thermodynamic Link (Boltzmann's Constant)}

\textbf{Constant \& Value:} \textbf{Boltzmann's constant} k\_B =
1.380649 × 10\^{}-23 J/K (exact, by definition)\hspace{0pt}. This
constant converts energy (joules) to temperature (kelvins), effectively
setting the scale of thermal energy per degree of freedom per Kelvin.

\textbf{Physical Meaning:} At 6D, the TORUS recursion transitions from
the realm of single particles and quantum interactions to the realm of
\textbf{many-particle systems and statistics}. Boltzmann's constant
introduces the concepts of temperature and entropy, marking the
emergence of \textbf{thermodynamics} in the hierarchy\hspace{0pt}. In
essence, by including k\_B, TORUS acknowledges that when enough degrees
of freedom accumulate (large numbers of particles), we need a way to
describe average energies, distributions, and thermal behavior. The 6D
constant provides the bridge: it links a microscopic energy scale (the
joule) to the macroscopic idea of temperature. Physically, this means
that at 6D, one can start talking about systems not just in terms of
individual quantum events, but in terms of ensemble properties like
\textbf{temperature (T)}, \textbf{entropy (S)}, and probability
distributions of states. It's the point where the model begins to
incorporate the second law of thermodynamics and statistical mechanics
as fundamental rather than derived.

\textbf{Relationships and Scale Harmony:} Boltzmann's constant works
closely with the 5D constant h to unify quantum and thermal scales. A
striking relationship is obtained by equating a single quantum of energy
to thermal energy: h ν = k\_B T. If we choose ν = 1/tₚ (the highest
fundamental frequency), we get T = h/(k\_B tₚ). Plugging in values, T \approx
8.9 × 10\^{}31 K\hspace{0pt}. This is on the order of 10\^{}32 K, which
is basically the \textbf{Planck temperature} (the 10D constant. In other
words, using the fundamental time scale (1D), the quantum of action
(5D), and Boltzmann's constant (6D) together naturally produces the
extreme unification temperature at 10D. This three-constant interplay is
a powerful confirmation that TORUS's constants are self-consistent
across scales: the \emph{h} and \emph{k\_B} introduced at 5D and 6D are
precisely such that when applied to the smallest time scale 1D, they
yield the highest meaningful temperature 10D\hspace{0pt}. Adjacent to
6D, we also have the next constant 7D (Avogadro's number) such that k\_B
combined with N\_A will yield the ideal gas constant R (8D)\hspace{0pt}.
Thus, k\_B is part of a \textbf{layering}: 5D (quantum) → 6D
(single-particle thermal) → 7D (Avogadro, turning single-particle to
per-mole). Below 6D, physics was about individual particles or quanta;
at 6D and beyond, we consider huge numbers of particles. Including k\_B
ensures that as soon as we consider ensembles, we have the correct
scaling to relate energy per particle to temperature. It effectively
seeds the recursion with the concept of \textbf{thermal energy per
degree of freedom}, allowing higher dimensions to build on full
statistical and thermodynamic laws. By 6D, each new layer is now summing
over vast numbers of states (whereas 5D and below dealt with one state
or a few). In summary, Boltzmann's constant is the keystone for moving
from quantum physics to classical thermodynamics within TORUS -- it
quantifies the point where averaging over many quanta becomes
fundamental.

\textbf{7D -- Collective Quantity (Avogadro's Number)}

\textbf{Constant \& Value:} \textbf{Avogadro's number} N\_A = 6.02214076
× 10\^{}23 (dimensionless count of particles per mole)\hspace{0pt}. This
is an exact defined number that sets the scale of one ``mole'' of
substance.

\textbf{Physical Meaning:} At 7D, the recursion introduces a standard
\textbf{large number of particles} as a single unit. Avogadro's number
is essentially the scaling factor between the microscopic world
(individual atoms/molecules) and the macroscopic world (bulk quantities
of matter in moles and grams)\hspace{0pt}. By including N\_A, TORUS
explicitly integrates \textbf{chemistry and bulk matter} into its
hierarchy. It means the model now has a built-in way to talk about, say,
6.022×10\^{}23 atoms of carbon (which is 12 grams) as a natural unit.
This level is where the idea of a ``mole'' -- a bridge between atomic
mass units and laboratory-scale masses -- becomes fundamental. In
physical terms, 7D marks the point of \emph{collective quantization} of
matter: instead of counting 1 particle, we count in units of Avogadro's
number of particles. This signals that TORUS at 7D is now addressing
phenomena of bulk matter, where sheer numbers of constituents are
themselves an important parameter.

\textbf{Inter-scale Connectivity:} Immediately, we see a beautiful
relationship: the 7D constant N\_A multiplied by the 6D constant k\_B
yields the 8D constant R (ideal gas constant)\hspace{0pt}. That is, N\_A
· k\_B = R, the constant that appears in the ideal gas law PV = N\_A
k\_B T = R T (per mole). In TORUS, this is \textbf{not coincidental} --
it's an explicit demonstration of recursion layering: the constant
introduced at one level (Avogadro) times the previous level's constant
(Boltzmann) produces the next level's constant (gas
constant)\hspace{0pt}. This harmonic progression underscores that once
we decide to include a ``per mole'' scaling, it naturally completes the
thermodynamic constants set. Additionally, Avogadro's number allows
conversion between the Planck mass scale and macroscopic masses: for
example, \emph{m}ₚ × N\_A \approx 1.31×10\^{}16 kg\hspace{0pt}, which is about
the mass of a small asteroid. While that particular product may not
signify a fundamental law, a more tangible one is that one mole of
protons (N\_A protons) has a mass of \textasciitilde1 gram (since 1
proton \textasciitilde1 atomic mass unit by definition, and 1 u × N\_A =
1 gram). This illustrates how N\_A serves as the link between the atomic
mass scale and the gram scale\hspace{0pt}. In the recursion context, 7D
sits between the microscopic constants (like h, k\_B) and the truly
macroscopic/cosmic constants (like G at 9D). It's the \emph{step that
explicitly brings large-N into play}. With N\_A, the theory can smoothly
talk about the energy of a mole of photons or the entropy in a mole of
gas, etc., which is essential for connecting to macroscopic
thermodynamics and even astrophysics. In summary, Avogadro's number in
TORUS emphasizes that \textbf{no scale is left out} -- by this stage,
the framework has spanned from Planck units up to human-scale units in a
continuous thread\hspace{0pt}. The presence of N\_A signals that the
recursion has grown from single particles to huge collections, setting
the stage for even larger structures and forces to come.

\textbf{8D -- Thermodynamic Completion (Ideal Gas Constant R)}

\textbf{Constant \& Value:} The \textbf{ideal gas constant} R =
8.314462618 J/(mol·K) (exact, being N\_A × k\_B)\hspace{0pt}. TORUS
assigns R as the characteristic constant of the 8D level.

\textbf{Physical Meaning:} By 8D, the set of constants needed to
describe \textbf{bulk matter thermodynamics} is complete. R is the
constant that appears in the ideal gas law PV = R T (for one mole of
gas), linking pressure, volume, and temperature for macroscopic amounts
of matter\hspace{0pt}. In the TORUS hierarchy, introducing R signifies
that we now have all the tools to describe a \textbf{classical,
continuum chunk of matter} (one that has volume, temperature, pressure,
and quantity), without yet invoking gravity. Essentially, 8D is the
capstone of internal thermodynamic description -- it encapsulates the
equation-of-state behavior of matter in aggregate. At this stage, TORUS
can account for systems like a gas in a container or heat flow in
materials purely from fundamental constants (now that R is included).
This level bridges the microscopic world (governed by k\_B and quantum
effects) and the cosmic-scale physics that comes next.

\textbf{Recursive Derivation:} As noted, R is \emph{not} an independent
constant in TORUS; it is literally the product of 6D and 7D constants: R
= N\_A · k\_B\hspace{0pt}. This direct derivation highlights the layered
construction of the hierarchy -- 8D emerges naturally once 6D and 7D are
in place. The presence of R allows us to easily move between
per-particle and per-mole descriptions. For example, a thermal energy of
k\_B T per particle corresponds to an energy of R T per mole. With R,
one can compute meaningful macroscopic energies: R × 300 K \approx 2.5×10\^{}3
J per mole (around room-temperature thermal energy per mole), or R ×
10\^{}9 K \approx 8.3×10\^{}9 J per mole (on the order of nuclear binding
energy per mole)\hspace{0pt}. These show that by using R we can quantify
chemistry (kJ per mole) and even nuclear processes in a unified way. R
also subtly ties into earlier constants in blackbody radiation and
astrophysical formulas: while not fundamental in those, R's constituents
(N\_A, k\_B) are present in derivations of the Stefan--Boltzmann
constant and other relations\hspace{0pt}. The key adjacent jump after 8D
is 9D -- the introduction of gravity. It's noteworthy that even before
explicitly introducing gravity, R allows some interplay with it: for
instance, in planetary atmospheres, the scale height H = R T/(M g)
involves R (thermodynamics) and g (gravity) together\hspace{0pt}. This
shows that at the 8D→9D boundary, matter's internal pressure (via R and
T) meets gravitational pull (via G giving weight \emph{mg}). Indeed,
phenomena like the \textbf{Jeans criterion} for gravitational collapse
involve both R (through temperature pressure support) and G (pulling
matter together), foreshadowing the integration at higher dimensions. To
summarize, 8D's ideal gas constant represents the \textbf{completion of
the thermodynamic toolkit} in TORUS. It signals that the theory now
fully accounts for bulk matter behavior in the absence of gravity, and
sets the stage to move to scales and forces that shape planets, stars,
and the universe as a whole.

\textbf{9D -- Gravity Introduction (Newton's Gravitational Constant G)}

\textbf{Constant \& Value:} \textbf{Newton's gravitational constant} G \approx
6.6743 × 10\^{}-11 m\^{}3·kg\^{}-1·s\^{}-2\hspace{0pt}. This constant
determines the strength of gravity in Newton's law (and enters general
relativity as well). TORUS assigns G as the fundamental constant of the
9D level.

\textbf{Physical Meaning:} At 9D, the recursion includes
\textbf{gravity} -- the first force that dominates at large, cosmic
scales. Introducing G marks a dramatic phase change in the hierarchy:
prior to this, the constants dealt with quantum forces (like
electromagnetism via \alpha, quantum action \hbar) and thermodynamic/statistical
behavior. With 9D, \textbf{astronomical and cosmological structures}
come into play\hspace{0pt}. G is the constant that allows matter to
clump into planets, stars, and galaxies, as it quantifies the
gravitational attraction between masses. In TORUS, the 9D stage means
the framework can now describe spacetime curvature and gravitational
binding -- phenomena like orbits, gravitational potential, and
eventually the expansion of the universe (via the Friedmann equations)
become accessible. Essentially, 9D is where the model gains the ability
to explain why the matter (described up to 8D) organizes into the
large-scale structures we observe, rather than remaining a diffuse gas.

\textbf{Consistency and Integration:} One might think gravity's strength
is independent, but in the Planck unit system, G is intertwined with
other constants. A revealing relationship from Planck units is:
G=c3tPmPG = \textbackslash frac\{c\^{}3
t\_P\}\{m\_P\}G=mP\hspace{0pt}c3tP\hspace{0pt}\hspace{0pt}\hspace{0pt}.
Plugging in the Planck time (1D), Planck mass (3D), and light speed (4D)
yields the observed G (this is essentially derived from tP=ℏG/c5t\_P =
\textbackslash sqrt\{\textbackslash hbar
G/c\^{}5\}tP\hspace{0pt}=ℏG/c5\hspace{0pt} and mP=ℏc/Gm\_P =
\textbackslash sqrt\{\textbackslash hbar
c/G\}mP\hspace{0pt}=ℏc/G\hspace{0pt}). Rearranged, it shows that once
\emph{t}ₚ, \emph{m}ₚ, and \emph{c} are set, G is \textbf{fixed by
consistency}\hspace{0pt}. Indeed, if we require that 1D, 2D, 3D, 4D
constants produce a coherent set of Planck units, G cannot be anything
else -- it is determined such that the combination G⋅tP2/\ellP3=1/c2G
\textbackslash cdot t\_P\^{}2/\ell\_P\^{}3 =
1/c\^{}2G⋅tP2\hspace{0pt}/\ellP3\hspace{0pt}=1/c2 (or similar dimensionless
unity conditions) holds\hspace{0pt}. TORUS incorporates this by not
treating G as arbitrary: by the time we ``turn on'' gravity at 9D, its
value is already harmonically related to the lower
constants\hspace{0pt}. In simpler terms, the prior recursion steps
``choose'' G such that the boundary between quantum and gravity (the
Planck scale) lines up exactly\hspace{0pt}-- which mirrors how nature's
Planck units are defined. With G now in play, we can examine
cross-links: for instance, combining G with earlier constants yields
enlightening scales. We saw one with \emph{m}ₚ (where G ties quantum
length to gravitational radius). Another is combining G with k\_B and
other constants: e.g., using G with the Planck temperature (10D) and
Boltzmann's constant relates to Planck mass as k\_B T\_P = m\_P c\^{}2,
implicitly involving G\hspace{0pt}. At 9D's introduction, gravity also
begins to interplay with thermodynamics: consider the \textbf{Jeans
length} for collapse of a gas cloud, λ\_J \textasciitilde{} √(R T/(G
ρ)). This critical length involves G (gravity) and R (8D thermodynamics)
together\hspace{0pt}. It shows that whether a cloud will collapse
(gravity wins) or disperse (pressure wins) depends on a balance between
8D and 9D constants. Thus, as soon as G enters, it starts linking with
the constants of matter and heat to govern structure formation. Finally,
note that 0D and 9D can be contrasted: 0D gave a dimensionless coupling
for microscopic force, and 9D gives the coupling for the
\textbf{macroscopic force}. The gravitational coupling constant for two
elementary particles (like two electrons) is incredibly small
(\textasciitilde10\^{}-40), reflecting gravity's relative weakness, but
when large masses are involved, G accumulates effect. TORUS highlights
that once G is introduced, the recursion can extend to explain why the
cosmos has galaxies and not just gas -- \textbf{structure emerges}. In
summary, 9D's gravitational constant is the gateway to cosmic physics in
TORUS, and it is carefully chosen to mesh with the tiny-scale constants
so that the entire range from quantum to cosmos remains self-consistent.

\textbf{10D -- Extreme Unification Temperature (Planck Temperature)}

\textbf{Constant \& Value:} The \textbf{Planck temperature} T\_P,
approximately 1.4168 × 10\^{}32 K\hspace{0pt}. This is the temperature
corresponding to the Planck energy (\textasciitilde2 × 10\^{}9 J per
particle) when divided by k\_B. TORUS uses T\_P as the fundamental
constant at 10D.

\textbf{Physical Meaning:} The 10D constant represents the
\textbf{highest energy density/temperature} of the current physical
cycle. Around 10\^{}32 Kelvin is the scale at which our known physics
likely ceases to be valid -- all quantum fields would be extremely
excited and gravitation becomes fully quantum. In cosmology, such a
temperature would have existed approximately 10\^{}-43 seconds after the
Big Bang (the Planck time) in conventional scenarios. TORUS treats 10D
as the point where \textbf{all forces unify into one}: at this ultimate
temperature, distinctions between the fundamental forces (strong,
electroweak, gravity) blur, and we have a symmetric state of
physics\hspace{0pt}. In essence, T\_P is like a capstone of energy in
the universe -- heating beyond this (or equivalently going to smaller
scales than \ell\_P or earlier than t\_P) is not meaningful within the
model, as it would require a new cycle or new physics. Thus, 10D marks
the \textbf{end of the line for increasing energy} in one TORUS cycle;
it's the point at which the recursion in energy terms is complete, and
any further ``increase'' would loop back (starting a new torus).

\textbf{Derivation and Cross-Scale Links:} Planck temperature is derived
directly from lower constants: by definition, k\_B T\_P = E\_P = m\_P
c\^{}2\hspace{0pt}. Substituting the Planck mass (3D), c (4D), and k\_B
(6D) gives T\_P \approx 1.4×10\^{}32 K\hspace{0pt}. This shows that the 10D
constant is not independent at all -- it's a \textbf{synthesis of 3D,
4D, 6D (and implicitly 5D and 9D)}\hspace{0pt}. In deriving m\_P we used
\hbar and G, so those are in the mix as well; thus T\_P encapsulates \hbar (5D),
G (9D), c (4D), and k\_B (6D) all in one number\hspace{0pt}. This
remarkable unity means 10D's value reflects the combined effect of
quantum mechanics, relativity, gravity, and thermodynamics. Adjacent
constants highlight its role: coming from 9D, without G setting m\_P, we
wouldn't get this extreme temperature value -- gravity's inclusion fixed
T\_P. And looking forward, 11D is about the unified force coupling which
conceptually ``kicks in'' at this temperature. In other words, 10D
provides the \textbf{energy scale} (temperature) at which unification
happens, and 11D will provide the \textbf{coupling strength} at that
unification\hspace{0pt}. One can view T\_P as the threshold at which our
cycle's laws must \textbf{restart or recycle}. TORUS suggests that once
this temperature is reached (e.g. at the end of a collapsing universe or
start of a Big Bang), a phase transition or ``bounce'' occurs that
effectively resets the universe's conditions -- akin to closing the
torus and opening a new one\hspace{0pt}. As a check, current physics
gives context: T\_P is vastly higher than any temperature achieved or
expected in stars or accelerators (it's billions of times hotter than
the center of a supernova, for instance). It's truly a
\textbf{theoretical upper limit} of temperature. By including it, TORUS
ensures that the model accounts for the earliest moments of the universe
and the potential unity of forces, rather than leaving that as an
open-ended infinity. In summary, 10D's Planck temperature is the
\textbf{culmination of energy scales} in the theory -- a unification
point derived from the interplay of all earlier constants, beyond which
a new cycle of physics begins.

\textbf{11D -- Unified Force Coupling (Dimensionless \textasciitilde1)}

\textbf{Constant \& Value:} The \textbf{unified coupling constant}
\alpha\textless sub\textgreater unified\textless/sub\textgreater, a
dimensionless number on the order of 1\hspace{0pt}. TORUS sets the 11D
constant essentially to 1 (within order of magnitude), representing the
strength of a hypothetical single force in the fully unified regime. In
other words, at this stage all fundamental forces have merged and are
characterized by one coupling parameter, which we take to be \alpha\_unified
\approx 1 for normalized units.

\textbf{Physical Meaning:} By 11D, we imagine the universe at an extreme
state of \textbf{symmetry and unification}. Having surpassed the Planck
temperature at 10D, the distinctions between electromagnetic, weak,
strong, and gravitational forces vanish; there is effectively
\textbf{one force} and one coupling describing interactions\hspace{0pt}.
The 11D constant thus represents the \textbf{pinnacle of unification} in
TORUS Theory -- all separate interaction constants have flowed together
into a single dimensionless constant. Setting it to \textasciitilde1 is
a matter of convention (one can always choose units at that scale so
that the coupling is unity), but it reflects the idea that at the
unification scale, the interaction is ``of order one,'' not feeble like
electromagnetism at low energy nor insanely weak like gravity between
elementary particles. Physically, this could correspond to a Grand
Unified Theory (GUT) state or something even beyond, where perhaps all
particles are identical or in a single super-multiplet due to symmetry
restoration\hspace{0pt}.

\textbf{Role in Recursion and Closure:} In the TORUS cycle, 11D serves
as a \textbf{reset point} before transitioning to the final
geometric/cosmological stages. Because \alpha\_unified is dimensionless, it
provides a pure number that can tie together all the dimensionless
ratios accumulated from 0D up to 10D. One way to see its importance: The
small coupling we started with at 0D (\alpha \textasciitilde1/137) has grown
(or ``run'') through various scales. By 11D, that growth results in a
coupling \textasciitilde1. In essence, the product of various scaling
factors from each level has taken 0.0073 and yielded
\textasciitilde1\hspace{0pt}. This is a strong consistency check: it
means the vast range of scales and strengths in the universe are chosen
such that when multiplied appropriately, they give unity at the
unification point. It ``closes the loop'' on strengths: the cycle began
with a tiny coupling and ends with a large coupling, ready to feed into
the next steps of cosmic structure\hspace{0pt}. In fact, TORUS posits
that 11D's unified force state effectively becomes the \emph{seed} for
the next cycle's early geometric conditions -- one can think of 11D as
analogous to 0D but at the opposite end of scale\hspace{0pt}. After
forces unify at 11D, what follows (12D and 13D) are the large-scale
structure constants (universe size and time) that \emph{complete} the
cycle and lead back to a new 0D. Thus, \alpha\_unified \textasciitilde1 is
like saying: ``if you multiply the inverse of the 0D coupling
(\textasciitilde137) by all the appropriate ratios up to this point, you
get \textasciitilde1.'' It ensures that no large disparity is left
unaccounted for by the time we have one force -- everything has been
balanced out.

In known physics, we don't yet have experimental confirmation of a
single unified coupling \textasciitilde1, but theoretical extrapolations
(with supersymmetry, for example) suggest the electroweak and strong
forces' couplings converge to a number not too far from unity at
\textasciitilde10\^{}16 GeV (the GUT scale)\hspace{0pt}. Including
gravity at \textasciitilde10\^{}19 GeV (Planck scale) is conjectural,
but TORUS essentially assumes such a convergence does happen. By baking
\alpha\_unified \approx1 into the hierarchy, the theory asserts that the
\textbf{unification is achieved within one cycle}, and we don't need an
external energy or scale beyond the 14D loop to bring forces together.
In summary, 11D's unified coupling constant is a \textbf{unitless
linchpin} of TORUS's self-consistency: it signifies that after
traversing an immense range of scales from 0D to 10D, the strengths of
nature's interactions coalesce into a single value, preparing the way
for the final cosmic-scale steps and the closure of the toroidal
universe.

\textbf{12D -- Cosmic Spatial Scale (Observable Universe Size)}

\textbf{Constant \& Value:} \textbf{Cosmic length scale} L\_U, on the
order of 4 × 10\^{}26 m\hspace{0pt}. This is roughly the radius of the
observable universe (\textasciitilde46 billion light years). TORUS takes
L\_U as a fundamental constant at 12D, representing the large-scale
spatial extent of the universe for this cycle.

\textbf{Physical Meaning:} At 12D, the recursion returns to a length
scale -- but at the \textbf{opposite extreme} from 2D's Planck length.
L\_U is essentially the size of the universe (or the horizon distance)
in the present cycle\hspace{0pt}. One can think of it as the
``diameter'' or ``circumference'' of the torus if we visualize the 14D
cycle as a closed loop in spacetime\hspace{0pt}. By including a
cosmic-length constant, TORUS integrates cosmology directly into the
fundamental framework: instead of treating the size of the universe as
just an initial condition or a result of dynamic evolution, it's
enshrined as a parameter that must align with all others. In effect, 12D
gives a \textbf{boundary (without boundary)} -- it's the largest
distance that fits in one cycle of the universe. Beyond this scale, one
might conceptually step into the next ``cell'' of the multiverse or wrap
around due to the toroidal topology. Physically, L\_U is related to the
distance light has traveled since the Big Bang, taking into account
cosmic expansion. It's the scale at which we have no further information
because light (or any causal influence) couldn't have reached us from
beyond that distance in the age of the universe.

\textbf{Harmonization with Other Scales:} One striking relation is
between 12D and 2D: multiply the smallest length by the largest length,
\emph{\ell}ₚ × L\_U. Using \ellₚ \textasciitilde1.6×10\^{}-35 m and L\_U
\textasciitilde4×10\^{}26 m gives \textasciitilde6.4×10\^{}-9, a tiny
dimensionless number (\textasciitilde10\^{}-8)\hspace{0pt}. While not
exactly unity, this number is far larger than, say, 10\^{}-60 (which one
would get if the universe were enormously bigger compared to the Planck
scale). TORUS notes that by including other factors like the 0D coupling
and the unified coupling, one might bring this product closer to
1\hspace{0pt}. The point is that the \textbf{disparity between micro and
macro lengths} in the TORUS universe is not completely arbitrary -- it
is tuned such that the extremes are related by the dynamics of the
cycle\hspace{0pt}. Another direct closure relation: the 13D time
constant \emph{T}\textless sub\textgreater U\textless/sub\textgreater{}
times \emph{c} (4D) yields a distance \textasciitilde1.3×10\^{}26 m,
which is on the same order as L\_U\hspace{0pt}. Indeed, c×TU\approxLUc
\textbackslash times T\_U \approx L\_Uc×TU\hspace{0pt}\approxLU\hspace{0pt} to
within a factor of order unity, which is exactly what we expect for an
almost flat, horizon-limited universe. This 12D--13D link is a
\textbf{cosmic echo} of the 1D--2D link (c × t\_P = \ell\_P), but at the
largest scale\hspace{0pt}. It signifies that space and time once again
correlate: the size of the universe is roughly what light could travel
in its age. Additionally, 12D is related to 9D (G) and the matter
content of the universe through cosmological equations. For example, the
Hubble length c/H0 (which is of order L\_U) depends on G and the average
density via H0∼GρH\_0 \textbackslash sim \textbackslash sqrt\{G
\textbackslash rho\}H0\hspace{0pt}∼Gρ\hspace{0pt} in the Friedmann
equation for a matter-dominated universe\hspace{0pt}. If one plugs in
the observed density, one gets a timescale on the order of the
universe's age, and hence a length scale on order 10\^{}26 m, showing
that \textbf{G and cosmic density ``choose'' L\_U} so that the
universe's size is consistent with its mass content. TORUS emphasizes
that 12D's value is fixed by the requirement of recursion closure and
consistency with observed cosmology\hspace{0pt}. By introducing 11D's
dimensionless unity prior, we had the freedom to incorporate a large
length without breaking scale consistency -- effectively 11D's ``1'' can
scale lengths or times without needing a new physics
constant\hspace{0pt}. Adjacently, 13D will provide the corresponding
time. Summing up, 12D in TORUS is the \textbf{cosmic horizon scale}
turned into a constant. It reflects the idea that the universe's vast
size is not just a random outcome but a part of a self-consistent
scheme: the smallest and largest lengths in nature are related through
the closed recursion, and the inclusion of L\_U ensures the model spans
a \textbf{complete range of scales from 10\^{}-35 m to 10\^{}26 m in one
cycle}.

\textbf{13D -- Cosmic Time Scale (Universe Age / Cycle Duration)}

\textbf{Constant \& Value:} \textbf{Cosmic time (universal cycle
duration)} T\_U, approximately 4.35 × 10\^{}17 s\hspace{0pt}, which is
about 13.8 billion years. TORUS takes T\_U as the fundamental time scale
of the 13D level, essentially the age of the universe (or the time from
Big Bang to the present closure point).

\textbf{Physical Meaning:} 13D provides the \textbf{temporal extent of
the entire universe's cycle}. In a standard cosmology context, this is
the time elapsed since the Big Bang. In TORUS, it can be thought of as
the duration of one full cycle of the toroidal recursion -- from the
initial 0D seed through the expansion and evolution up to the present,
possibly ending in a turnaround or ``closure'' event\hspace{0pt}.
Including T\_U as a constant means TORUS treats the age of the universe
not just as a measured historical fact, but as a parameter that is
determined by the interplay of fundamental physics (much like c or G).
It implies the universe's longevity is \textbf{built into the theory}
and must be consistent with all other constants, rather than being an
arbitrary initial condition\hspace{0pt}. In a cyclic or closed universe
picture, T\_U might also represent the time until a recollapse or
bounce, after which a new cycle begins. Thus, 13D marks the
\textbf{completion of the time dimension's loop} -- after this much
time, the recursion is supposed to ``reset'' in the TORUS model, feeding
13D's output back into 0D.

\textbf{Relations and Closure:} We already noted the essential relation
c × T\_U \approx L\_U\hspace{0pt}. Numerically, 4.35×10\^{}17 s × 3×10\^{}8
m/s \approx 1.3×10\^{}26 m, which is on the same order as our L\_U
\textasciitilde4×10\^{}26 m (a factor difference of a few is acceptable
given cosmic expansion and model specifics)\hspace{0pt}. This is exactly
what one expects: the horizon distance is c times the universe age
(adjusted for expansion). This relation is a \textbf{consistency check}
that at the largest scale, space and time are in sync, just as they were
at the smallest scale (c ties t\_P to \ell\_P). It essentially says that in
one universe-lifetime, light can traverse the universe -- a necessary
condition for the toroidal closure (no causally disconnected
pieces)\hspace{0pt}. The 13D constant also ties in with gravity and the
content of the universe: using the Friedmann equation for a flat
matter-dominated universe, one finds TU∼23H0-1\approx231GρT\_U
\textbackslash sim \textbackslash frac\{2\}\{3\} H\_0\^{}\{-1\} \approx
\textbackslash frac\{2\}\{3\}
\textbackslash sqrt\{\textbackslash frac\{1\}\{G
\textbackslash rho\}\}TU\hspace{0pt}∼32\hspace{0pt}H0-1\hspace{0pt}\approx32\hspace{0pt}Gρ1\hspace{0pt}\hspace{0pt}\hspace{0pt}.
This shows T\_U depends on G (9D) and the average density ρ (which
itself is set by things like particle masses, cosmological parameters,
etc., ultimately traceable to earlier constants). In fact, 13D encodes a
combination of G (9D), R (8D, through the equation of state of cosmic
components), and even \alpha (0D) through astrophysical
processes\hspace{0pt}. For example, the tiny 0D coupling \alpha influences
nuclear reaction rates in the early universe, determining how much
hydrogen and helium form, which in turn affects the matter density and
thus the expansion rate and age. TORUS points out that such multi-scale
links mean the \textbf{microscopic physics can influence the cosmic
timetable}. The enormous ratio T\_U/t\_P (\textasciitilde8×10\^{}60) can
be factorized into contributions from various fundamental ratios: indeed
\textasciitilde10\^{}60 \approx 10\^{}2 × 10\^{}38 × 10\^{}20 was
noted\hspace{0pt}, corresponding to (approximately) the inverse of \alpha
(∼10\^{}2), times the inverse gravitational coupling of an electron
(∼10\^{}38), times an entropy or particle-number factor (∼10\^{}20). The
fact that these numbers multiply to the observed age in Planck units
hints that the values of \alpha, G (as it affects particle masses), and the
number of particles in the universe (entropy) are all related in a way
that yields the universe's age -- a kind of large-number coincidence
that TORUS elevates to a principle rather than a fluke\hspace{0pt}. In
the recursion, 13D's adjacent link to 12D was the cT\_U \approx L\_U closure;
looking beyond 13D, there is no ``14D'' with new physics, but rather the
idea that after T\_U, the universe's state transitions into the starting
conditions for a new cycle (0D)\hspace{0pt}. This could correspond to a
Big Crunch followed by a bounce or some reset mechanism -- the
\textbf{toroidal closure} in time. Thus, 13D not only quantifies our
universe's lifetime but also ensures the cycle is a loop: once this time
passes, we circle back to a 0D-like origin for the next iteration.

In summary, the 13D cosmic time constant is the \textbf{culmination of
the TORUS hierarchy}: it places the universe's age on the same
fundamental footing as the speed of light or Planck's constant. By doing
so, TORUS claims that even the large-scale parameters (size and duration
of the universe) are determined by the interplay of all smaller-scale
constants, achieving a deep coherence across all scales. After 13D, the
model's demand for self-consistency requires that we do not introduce
any new arbitrary scale -- instead, we recognize that the ``end'' feeds
into the ``beginning,'' completing the \textbf{eternal recursion} of the
TORUS universe\hspace{0pt}.

\end{document}
