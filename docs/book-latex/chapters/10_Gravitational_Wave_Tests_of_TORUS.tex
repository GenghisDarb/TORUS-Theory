% Options for packages loaded elsewhere
\PassOptionsToPackage{unicode}{hyperref}
\PassOptionsToPackage{hyphens}{url}
%
\documentclass[
]{article}
\usepackage{amsmath,amssymb}
\usepackage{iftex}
\ifPDFTeX
  \usepackage[T1]{fontenc}
  \usepackage[utf8]{inputenc}
  \usepackage{textcomp} % provide euro and other symbols
\else % if luatex or xetex
  \usepackage{unicode-math} % this also loads fontspec
  \defaultfontfeatures{Scale=MatchLowercase}
  \defaultfontfeatures[\rmfamily]{Ligatures=TeX,Scale=1}
\fi
\usepackage{lmodern}
\ifPDFTeX\else
  % xetex/luatex font selection
\fi
% Use upquote if available, for straight quotes in verbatim environments
\IfFileExists{upquote.sty}{\usepackage{upquote}}{}
\IfFileExists{microtype.sty}{% use microtype if available
  \usepackage[]{microtype}
  \UseMicrotypeSet[protrusion]{basicmath} % disable protrusion for tt fonts
}{}
\makeatletter
\@ifundefined{KOMAClassName}{% if non-KOMA class
  \IfFileExists{parskip.sty}{%
    \usepackage{parskip}
  }{% else
    \setlength{\parindent}{0pt}
    \setlength{\parskip}{6pt plus 2pt minus 1pt}}
}{% if KOMA class
  \KOMAoptions{parskip=half}}
\makeatother
\usepackage{xcolor}
\setlength{\emergencystretch}{3em} % prevent overfull lines
\providecommand{\tightlist}{%
  \setlength{\itemsep}{0pt}\setlength{\parskip}{0pt}}
\setcounter{secnumdepth}{-\maxdimen} % remove section numbering
\ifLuaTeX
  \usepackage{selnolig}  % disable illegal ligatures
\fi
\IfFileExists{bookmark.sty}{\usepackage{bookmark}}{\usepackage{hyperref}}
\IfFileExists{xurl.sty}{\usepackage{xurl}}{} % add URL line breaks if available
\urlstyle{same}
\hypersetup{
  hidelinks,
  pdfcreator={LaTeX via pandoc}}

\author{}
\date{}

\begin{document}

\textbf{Chapter 10: Gravitational Wave Tests of TORUS}

\textbf{10.1 Predicted Dispersion and Polarization Effects}

Gravitational waves in \textbf{General Relativity (GR)} propagate as
ripples in spacetime that travel at the speed of light with \emph{no}
frequency-dependent dispersion. In vacuum GR, all gravitational wave
frequencies move at the same speed (exactly \$c\$) and there are only
two allowed polarization modes -- the so-called ``plus'' and ``cross''
transverse tensor polarizations\hspace{0pt}. \textbf{Dispersion} refers
to a dependence of wave speed on frequency, which standard GR predicts
should not occur for gravitational waves. \textbf{Polarization} refers
to the orientation states of the wave's oscillations; GR's massless
spin-2 graviton permits exactly two independent polarization states in
four dimensions.

\textbf{TORUS modifications:} By introducing a \emph{structured
recursion through 14 dimensions (0D through 13D)}, TORUS Theory adds
subtle extra terms to the Einstein field equations (via
higher-dimensional feedback) that alter gravitational wave
propagation\hspace{0pt}. These recursion-induced terms lead to two key
anomalous effects that depart from GR's expectations:

\begin{itemize}
\item
  \textbf{Frequency-Dependent Speed (Dispersion):} TORUS predicts that
  gravitational waves may exhibit an extremely tiny frequency-dependent
  speed in vacuum, meaning higher-frequency components travel at a
  slightly different speed than lower-frequency components\hspace{0pt}.
  In practice, a short burst of gravitational waves (for example, from a
  neutron star merger) would not arrive perfectly ``in sync'' for all
  frequencies -- higher-frequency ripples could arrive marginally
  earlier or later than low-frequency ones. Quantitatively, the group
  velocity \$v\_g\$ might differ from \$c\$ by a fractional amount on
  the order of \$10\^{}\{-15\}\$--\$10\^{}\{-14\}\$ over astronomical
  distances for kilohertz-frequency waves\hspace{0pt}. (By comparison,
  multi-messenger observations of the neutron star merger GW170817,
  which had an observed gravitational wave and gamma-ray flash, have
  constrained any deviation of gravitational wave speed from \$c\$ to
  less than about one part in \$10\^{}\{15\}\$\hspace{0pt}. TORUS
  suggests a dispersion effect potentially just below that current
  bound, meaning it could become detectable as instruments improve.) In
  summary, unlike GR which predicts no dispersion, TORUS's framework
  implies a \textbf{measurable dispersion} of gravitational waves --
  albeit a minute effect -- as a direct consequence of its recursive
  structure.
\item
  \textbf{Additional Polarization Mode:} Alongside the usual plus and
  cross polarizations of GR, TORUS allows for a possible \textbf{extra
  polarization} component in gravitational waves\hspace{0pt}. This would
  manifest as a weak longitudinal or ``scalar'' polarization mode
  (sometimes described as a breathing mode) with an amplitude at roughly
  the \$10\^{}\{-3\}\$ (0.1\%) level relative to the standard tensor
  modes\hspace{0pt}. Such a polarization is forbidden in pure GR, which
  only permits two transverse modes, but extra polarizations can arise
  in extended gravity theories that include new degrees of freedom (for
  example, scalar-tensor or vector-tensor theories). In TORUS, the extra
  mode is tied to the higher-dimensional recursion effects --
  essentially, the 14D hierarchical structure introduces a small
  additional degree of freedom in the gravitational field equations.
  This might be correlated with large-scale geometric features of the
  recursion (for instance, a dependence on the source's orientation
  relative to the cosmic 13D recursion axis)\hspace{0pt}. The net result
  is that gravitational waves in TORUS could carry a tiny ``footprint''
  of the theory's extra structure: a faint polarization component beyond
  the plus/cross of GR. Detecting an anomalous polarization component in
  gravitational wave signals would be a striking signature of TORUS's
  recursive framework, because it would indicate a violation of GR's
  polarization prediction in exactly the manner (small
  scalar-longitudinal component) that TORUS permits\hspace{0pt}.
\end{itemize}

These two deviations -- slight dispersion and an extra polarization --
are \textbf{empirically testable}. The magnitude of the effects is
predicted to be very small (on the threshold of current detection
limits), but importantly, they provide concrete benchmarks. If observed,
they would lend strong support to TORUS by revealing new physics beyond
GR. If they are not observed when instruments are sensitive enough, that
absence can falsify or constrain TORUS (as discussed later). The key
point is that TORUS's recursive unification does not remain a purely
theoretical construct; it \emph{makes quantitative predictions} about
gravitational waves that distinguish it from standard physics, ensuring
the theory can be confronted with observational reality\hspace{0pt}.

\textbf{10.2 Experimental Sensitivity with LIGO, Virgo, LISA}

Modern gravitational wave detectors offer a powerful means to search for
the subtle effects predicted by TORUS. Here we discuss the capabilities
of the major observatories -- the ground-based \textbf{LIGO/Virgo
network} and the future space-based \textbf{LISA} -- and how they can
test TORUS's dispersion and polarization predictions. We consider the
sensitivity thresholds, detection methods, and specific observational
signatures that these experiments can utilize.

\textbf{Ground-Based Interferometers (LIGO, Virgo, KAGRA):} The Advanced
LIGO and Virgo detectors (along with KAGRA in Japan, and soon
LIGO-India) operate in the high-frequency band (tens to thousands of Hz)
and have already measured gravitational waves from multiple compact
binary mergers. These kilometer-scale interferometers are sensitive to
minute differences in the travel time and waveform of incoming
gravitational waves. Crucially, they have tested for deviations from GR
in gravitational wave propagation. For example, the LIGO/Virgo
observations of binary neutron star merger GW170817 found no significant
difference between the arrival time of gravitational waves and the
speed-of-light signal, placing an upper bound on any speed variation of
order \$10\^{}\{-15\}\$ (fractional) or less\hspace{0pt}. Similarly,
LIGO and Virgo data analyses so far have not revealed any dispersion in
the waveforms -- any frequency-dependent arrival time differences are
below the detection threshold \textasciitilde10\^{}(-15)\hspace{0pt}.
They have also looked for non-standard polarization components by
comparing signals across the global detector network. So far, all
observed signals have been consistent with the two tensor polarizations
of GR, with no obvious requirement for an extra polarization mode
(within current sensitivity limits). These results already
\textbf{constrain TORUS's effects}, indicating that if TORUS's predicted
dispersion and scalar polarization exist, they must be at or below the
current detection limits (\textasciitilde10\^{}−15 in speed fraction,
and \textasciitilde0.1\% in amplitude). The good news for TORUS is that
these detectors are still improving, and the effects could lie just
beyond present capabilities\hspace{0pt}. The strategy for ground
interferometers to detect TORUS anomalies involves precision timing and
waveform analysis: by examining high signal-to-noise events and looking
for frequency-dependent phase shifts (for dispersion) or anomalies in
the pattern of detector responses (for polarization), any small
deviations from GR can be teased out. For instance, if a future binary
neutron star merger signal (``chirp'') shows that the highest-frequency
part of the waveform arrives slightly earlier or later than expected
under dispersionless propagation, that would be evidence of
gravitational wave dispersion. Likewise, with multiple detectors
oriented differently (LIGO Hanford and Livingston in the US, Virgo in
Europe, KAGRA in Asia, etc.), the network can decompose the polarization
content of incoming waves. A consistent residual signal that cannot be
explained by a combination of plus/cross polarizations -- for example,
an in-phase strain seen equally by all detectors regardless of
orientation -- could indicate the presence of a scalar-longitudinal
mode. The addition of new detectors (like LIGO-India in the near future)
will improve the sky coverage and polarization sensitivity of the
network, increasing the chances of catching a tiny polarization anomaly
if it exists\hspace{0pt}.

\textbf{Space-Based Interferometer (LISA):} The \textbf{Laser
Interferometer Space Antenna (LISA)}, planned for launch in the 2030s,
will consist of a triangular constellation of satellites separated by
millions of kilometers, sensitive to lower-frequency gravitational waves
(millihertz to 0.1 Hz). LISA's enormous baseline and the fact that it
will observe signals from distant, massive black hole mergers and other
cosmological sources make it exceptionally well-suited to probe minute
dispersion effects accumulating over vast distances\hspace{0pt}. In
TORUS's context, LISA could provide a decisive test of gravitational
wave dispersion: even a fractional speed difference of
\$10\^{}\{-15\}\$, which might be marginal in ground-based detectors
observing relatively nearby stellar-mass events, could become evident in
LISA's observation of a supermassive black hole binary merger billions
of light years away. Over such travel distances, a frequency-dependent
speed difference would cause a slight distortion in the wave packet --
high-frequency components might arrive noticeably earlier (or later)
than low-frequency ones, leading to a frequency-dependent phase shift in
the observed waveform. LISA's data analysis will therefore include
searches for deviations from the expected phase evolution of inspiral
signals. If a gravitational wave event observed by LISA shows that its
waveform cannot be simultaneously fit at all frequencies by the
assumption of a single speed \$c\$, that would signal a
\textbf{dispersion} consistent with TORUS's prediction\hspace{0pt}.
Additionally, LISA's design (a coherent three-arm detector in space)
allows it to measure polarization states of passing gravitational waves.
While LISA alone (with effectively two or three interferometer channels)
cannot fully distinguish all six possible polarization modes in a
general metric theory, it can test for the presence of modes beyond the
two tensor ones by looking at the specific pattern of signals in its
multiple arms. In combination with ground detectors (for sources that
produce signals in both bands) or by using the fact that a polarization
like a scalar mode would produce a distinctive breathing pattern on the
LISA constellation, LISA could also contribute to identifying extra
polarization components. In summary, LISA offers \textbf{extreme
sensitivity to TORUS effects}: its long-baseline measurement of wave
travel allows detection of tiny dispersion over cosmological distances,
and its multi-arm configuration can cross-check the polarization content
of low-frequency gravitational waves\hspace{0pt}.

\textbf{Observational scenarios and signatures:} To concretely
illustrate, consider a distant binary neutron star or black hole merger
observed in the 2030s. In TORUS's scenario, as the gravitational wave
passes through the intervening billions of light years, the
higher-frequency parts of the signal might get slightly out of sync due
to a recursion-induced dispersion. By the time the wave reaches Earth
(or LISA in space), the arrival times of various frequency components
are no longer perfectly aligned. Analysts would reconstruct the signal
and find, for example, that the early high-frequency ``chirp'' portion
of the waveform is fractionally delayed compared to what GR predicts
when extrapolated from the low-frequency part -- a discrepancy not
attributable to known matter effects (like dispersion from interstellar
plasma, which is negligible for gravitational waves)\hspace{0pt}. This
\textbf{frequency-dependent arrival lag} would be a hallmark of TORUS.
Meanwhile, the same event could be observed by a network of ground
detectors on Earth. If those detectors, with their different
orientations, record signals that cannot be explained by any combination
of two transverse polarizations, it might indicate an extra polarization
at play. For instance, suppose that after subtracting the best-fit
plus/cross waveform, a small residual signal of identical phase appears
in all detectors -- that could point to a longitudinal strain component
affecting all sites equally, consistent with a scalar polarization.
Seeing such a pattern repeatedly (even at the 0.1\% level in amplitude)
in multiple independent events would build confidence that a real new
polarization mode is present\hspace{0pt}. Both of these signatures -- a
slight time-frequency distortion of waveforms, and an anomalous
polarization signal in a network -- are within reach of upcoming
experiments. The advanced LIGO/Virgo network (with upgrades sometimes
termed ``LIGO A+'' and eventually next-generation observatories like
Cosmic Explorer or Einstein Telescope) will dramatically improve
sensitivity in the coming decade, and LISA will open a new observational
window. \textbf{TORUS's predictions have been framed to be testable by
these instruments}: the dispersion is predicted to be just beyond
current non-detection limits (so it \emph{could} appear with the next
order-of-magnitude sensitivity improvement), and the extra polarization
is small but not zero, meaning a dedicated search might uncover it if
present\hspace{0pt}. In effect, the experimental strategy is clear --
\emph{listen} for any slight frequency-dependent arrival effects in
gravitational wave chirps and \emph{look} for any polarization content
beyond GR's two modes. If TORUS is correct, then as detectors reach the
required precision, they should begin to see these tiny deviations
emerge against the otherwise precise predictions of GR.

\textbf{10.3 Defining Clear Empirical Falsifiability Conditions}

A cornerstone of scientific theory is \textbf{falsifiability} -- the
idea that a theory must make predictions that could, in principle, be
proven wrong by experiment or observation. In other words, there must
exist a possible outcome that contradicts the theory if the theory is
false. TORUS Theory explicitly embraces this principle: it is
constructed to be testable and at risk of falsification, rather than
being a merely philosophical or uncheckable framework\hspace{0pt}. By
formulating concrete predictions (such as the gravitational wave
dispersion and polarization effects above), TORUS provides clear
criteria by which nature can refute it. This commitment to empirical
accountability not only differentiates TORUS from some more speculative
``theories of everything,'' but also lends credibility -- it shows that
TORUS is willing to stake its validity on the outcome of real
measurements.

In the context of gravitational waves, we can \textbf{define specific
observational conditions that would falsify TORUS's predictions}. If
rigorous experiments fail to find the anomalies that TORUS anticipates
-- beyond the levels that TORUS could reasonably hide -- then the theory
would be contradicted. The following are clear falsifiability conditions
for TORUS in gravitational wave tests:

\begin{enumerate}
\def\labelenumi{\arabic{enumi}.}
\item
  \textbf{No Dispersion Detected to Exceedingly High Precision:} If
  gravitational waves are observed to propagate \emph{exactly} as in GR
  with no frequency-dependent speed differences down to a precision well
  beyond \$10\^{}\{-15\}\$, TORUS's predicted dispersion is ruled out.
  For example, suppose the LISA mission and future ground detectors
  analyze numerous distant merger events and find that high-frequency
  and low-frequency gravitational wave components arrive with timing
  differences consistent with zero to within, say, one part in
  \$10\^{}\{-16\}\$ or better. Such an observation would show that any
  vacuum dispersion must be an order of magnitude smaller than TORUS's
  minimum predicted effect (around
  \$10\^{}\{-14\}\$--\$10\^{}\{-15\}\$)\hspace{0pt}. In that scenario,
  the \textbf{absence of dispersion} at the sensitivities where TORUS
  expected a signal would directly falsify that aspect of the theory.
  TORUS would either have to significantly revise the recursion model to
  suppress any dispersion, or else the framework in its current form
  would be considered invalid. In short, if gravitational wave signals
  continue to show no frequency-dependent arrival time differences even
  as our timing measurements reach the
  \$10\^{}\{-16\}\$--\$10\^{}\{-17\}\$ range, it means the TORUS
  dispersion prediction fails empirically\hspace{0pt}.
\item
  \textbf{No Extra Polarization Observed (Within Tight Limits):} If all
  gravitational wave observations consistently show only the two
  standard tensor polarizations, with no trace of any additional mode
  even at the \$\textbackslash sim10\^{}\{-3\}\$ level or below, then
  TORUS's extra polarization prediction is falsified. Concretely,
  imagine that the expanded network of detectors (LIGO, Virgo, KAGRA,
  LIGO-India, and future observatories) examines a large sample of
  events and perhaps even a stochastic background, and finds that the
  data can be completely explained by two polarization components. If a
  dedicated search for a longitudinal/scalar polarization yields null
  results and places an upper bound on any such component of, say,
  \$10\^{}\{-4\}\$ of the signal (or tighter), this would undercut
  TORUS's expectation of a \$10\^{}\{-3\}\$ effect. For instance, the
  lack of any detectable signal in polarization channels beyond GR's two
  -- even with 10× to 100× improved sensitivity over current detectors
  -- would indicate that no third mode exists at the level TORUS
  requires\hspace{0pt}. Such a finding would be in direct conflict with
  the theory's prediction of a small but non-zero extra polarization.
  Therefore, \textbf{if no anomalous polarization is observed} as
  detector sensitivity and analysis techniques improve (approaching the
  fractional percentage level), TORUS's modified gravity framework would
  be strongly disconfirmed.
\end{enumerate}

Taken together, these conditions set a high bar that TORUS must clear to
survive as a viable theory. The \textbf{``pass/fail'' criteria are
unambiguous}: TORUS will be \emph{failed} if nature shows (within
experimental error) that gravitational waves have no dispersion and no
extra polarization to the precision that encompasses TORUS's predicted
values\hspace{0pt}. Notably, this is not an all-or-nothing one-shot
test; it's a matter of progressively tightening the bounds. With each
improvement in detector sensitivity, the allowable window for TORUS's
effects narrows. If after, say, a decade of LISA data and
next-generation ground observations, the dispersion fraction is
constrained at the \$10\^{}\{-16\}\$ level and no hint of a third
polarization is seen, the \textbf{recursive effect is essentially
absent} and TORUS would either have to abandon those predictions or be
considered falsified in its current form\hspace{0pt}. This kind of
outcome would mean that the recursion-driven modifications at the 9D
gravity level are far smaller than posited, undermining a key piece of
TORUS's unified framework\hspace{0pt}.

By contrast, if the predicted anomalies \emph{are} observed -- even
marginally at first, and then with increasing confidence -- it would
corroborate TORUS and validate the idea that subtle higher-dimensional
recursion influences are real. Importantly, \textbf{TORUS has set itself
up for genuine risk}: it made precise, testable statements that could
have turned out differently. This willingness to be tested is a hallmark
of scientific rigor. TORUS is not protected by untestability; it stands
to gain credibility if experiments agree, and to lose credibility (or be
discarded) if they don't\hspace{0pt}. In this way, outlining clear
empirical falsifiability conditions enhances the theory's standing -- it
shows that TORUS is formulated in the spirit of empirical science, where
nature has the final say. The coming years of gravitational wave
astronomy thus represent a critical proving ground for TORUS. Either the
theory's ``fingerprints'' (a slight dispersion and an extra
polarization) will be detected, lending strong support to the Recursive
Unified Framework, or the lack of any such evidence will serve as a
decisive reality check, potentially ruling out TORUS's gravitational
sector. \textbf{Either outcome is scientifically valuable}: we will have
tested a bold unified theory against the empirical truth of the cosmos,
thereby deepening our understanding of gravitational physics and the
foundations of reality. In sum, TORUS's engagement with gravitational
wave tests exemplifies the theory's empirical grounding -- it turns the
profound concepts of a 14-dimensional recursive universe into concrete
predictions that today's and tomorrow's experiments can confirm or
refute, which is exactly the standard any theory of everything must meet
to be taken seriously.\hspace{0pt}

\end{document}
