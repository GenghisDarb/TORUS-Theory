\PassOptionsToPackage{unicode=true}{hyperref} % options for packages loaded elsewhere
\PassOptionsToPackage{hyphens}{url}
%
\documentclass[]{article}
\usepackage{lmodern}
\usepackage{amssymb,amsmath}
\usepackage{ifxetex,ifluatex}
\usepackage{fixltx2e} % provides \textsubscript
\ifnum 0\ifxetex 1\fi\ifluatex 1\fi=0 % if pdftex
  \usepackage[T1]{fontenc}
  \usepackage[utf8]{inputenc}
  \usepackage{textcomp} % provides euro and other symbols
\else % if luatex or xelatex
  \usepackage{unicode-math}
  \defaultfontfeatures{Ligatures=TeX,Scale=MatchLowercase}
\fi
% use upquote if available, for straight quotes in verbatim environments
\IfFileExists{upquote.sty}{\usepackage{upquote}}{}
% use microtype if available
\IfFileExists{microtype.sty}{%
\usepackage[]{microtype}
\UseMicrotypeSet[protrusion]{basicmath} % disable protrusion for tt fonts
}{}
\IfFileExists{parskip.sty}{%
\usepackage{parskip}
}{% else
\setlength{\parindent}{0pt}
\setlength{\parskip}{6pt plus 2pt minus 1pt}
}
\usepackage{hyperref}
\hypersetup{
            pdfborder={0 0 0},
            breaklinks=true}
\urlstyle{same}  % don't use monospace font for urls
\usepackage{longtable,booktabs}
% Fix footnotes in tables (requires footnote package)
\IfFileExists{footnote.sty}{\usepackage{footnote}\makesavenoteenv{longtable}}{}
\setlength{\emergencystretch}{3em}  % prevent overfull lines
\providecommand{\tightlist}{%
  \setlength{\itemsep}{0pt}\setlength{\parskip}{0pt}}
\setcounter{secnumdepth}{0}
% Redefines (sub)paragraphs to behave more like sections
\ifx\paragraph\undefined\else
\let\oldparagraph\paragraph
\renewcommand{\paragraph}[1]{\oldparagraph{#1}\mbox{}}
\fi
\ifx\subparagraph\undefined\else
\let\oldsubparagraph\subparagraph
\renewcommand{\subparagraph}[1]{\oldsubparagraph{#1}\mbox{}}
\fi

% set default figure placement to htbp
\makeatletter
\def\fps@figure{htbp}
\makeatother

% Wrap all longtable environments in resizebox to prevent overflow
\let\oldlongtable\longtable
\let\endoldlongtable\endlongtable
\renewenvironment{longtable}{\begin{resizebox}{\textwidth}{!}{\oldlongtable}}{\endoldlongtable\end{resizebox}}

% --- BEGIN EQUATION FORMATTING FIXES ---
% Replace HTML-like sub/sup tags with LaTeX math mode
\newcommand{\subscript}[1]{\ensuremath{_{\mathrm{#1}}}}
\newcommand{\superscript}[1]{\ensuremath{^{\mathrm{#1}}}}
% Usage: $A\subscript{B}$ or $A\superscript{B}$
% --- END EQUATION FORMATTING FIXES ---

\date{}

\begin{document}

\textbf{OSQN Drift in a Quartz-Oscillator Loop}

\textbf{Lab Worksheet -- rev B-0.1 (bench-ready)}

\textbf{0 Mission Recap}

Probe the \textbf{Observer-State Quantum Number (OSQN)} prediction that
a \emph{sealed-loop} quartz reference will exhibit a tiny,
phase-coherent frequency ripple when an ``observer channel'' is
logically prepared---even if the channel is never activated.\\
TORUS says the ripple shows up as

Δf  ≈  ±  f014    (side-bands)and/orf˙∼10−6  f0\textbackslash{}Delta f
\textbackslash{};\textbackslash{}approx\textbackslash{};
\textbackslash{}pm\textbackslash{};\textbackslash{}frac\{f\_0\}\{14\}\textbackslash{};\textbackslash{};(\textbackslash{}text\{side-bands\})\textbackslash{}quad\textbackslash{}text\{and/or\}\textbackslash{}quad
\textbackslash{}dot f \textbackslash{}sim
10\^{}\{-6\}\textbackslash{};f\_0Δf≈±14f0​​(side-bands)and/orf˙​∼10−6f0​

within a χ-signature dwell window of 10--120 s after preparation.

\textbf{1 Bill of Materials (≈ US \$120 total)}

\begin{longtable}[]{@{}lll@{}}
\toprule
\textbf{Qty} & \textbf{Part} & \textbf{Notes / Spec}\tabularnewline
\midrule
\endhead
1 & \textbf{TCXO 10 MHz} (Abracon ASTX-13 or similar) \textbf{-or-}
watch-grade \textbf{32.768 kHz crystal + CMOS inverter} & The TCXO's
built-in oven removes most temp drift; watch crystal is cheaper but
needs good shielding.\tabularnewline
1 & \textbf{MCU board} (STM32 ``Nucleo-64'', Teensy 4.1, or RP2040) &
Capture-compare timer or hardware-PPS gate; 48 MHz+ clock
ideal.\tabularnewline
1 & \textbf{20 MHz logic analyzer / USB scope} (Saleae-type) & For raw
edge timing \& side-band FFT snapshots.\tabularnewline
1 & \textbf{Low-noise linear 3.3 V} supply (LT3045, ADM7150) & Avoid
switch-mode ripple into the oscillator.\tabularnewline
1 & \textbf{Faraday box or metal cookie-tin} + RF gasket tape & Optional
but recommended for baseline run.\tabularnewline
--- & SMA / BNC cables, breadboard or SMT adapter, thin PTFE wire & Keep
signal lines short (\textless{}5 cm) to reduce inductive
pickup.\tabularnewline
--- & \textbf{DS18B20} temperature probe (optional) & Correlate temp
drift if you skip the TCXO.\tabularnewline
\bottomrule
\end{longtable}

\emph{Everything above is vendor-agnostic; grab the closest equivalents
you have on hand.}

\textbf{2 Circuit Snapshot}

markdown

CopyEdit

+3V3 ──► TCXO 10 MHz ──► SMA tee

│

STM32 TIM2 CH1 ◄─────────────┘ (count edges)

│

Logic-analyzer CH0 ◄─────────┘

│

GND ─────────────────────────┘

Optional ``observer channel'':

TCXO PPS out ─► *not* connected (stub trace)

MCU pin X ─► high-Z input (logically configured)

\emph{If you use a 32 kHz watch crystal: build a Pierce oscillator
around a CMOS inverter (e.g., 74LVC1G04) and route the output exactly as
above.}

\textbf{3 Test-Run Matrix}

\begin{longtable}[]{@{}lllll@{}}
\toprule
\textbf{Run ID} & \textbf{Box Lid} & \textbf{``Observer channel'' prep}
& \textbf{Duration} & \textbf{Goal}\tabularnewline
\midrule
\endhead
\textbf{B-0} Baseline & Closed & \emph{OFF} (pin left floating, MCU
ignores) & 2 h & Establish Allan-dev \& PSD floor\tabularnewline
\textbf{B-1} Prepared & Closed & \emph{ON} (pin configured as digital
in, though nothing ever toggles) & 2 h & Look for χ ripple with latent
path\tabularnewline
\textbf{B-2} Dormant & Open & \emph{ON} & 1 h & Isolate EM /
human-proximity artefacts\tabularnewline
\textbf{B-3} Sham & Closed & \emph{OFF} but MCU toggles a dummy GPIO
elsewhere & 1 h & Guard vs. firmware-noise false positive\tabularnewline
\bottomrule
\end{longtable}

Repeat each run twice on different days if possible.

\textbf{4 Measurement Procedure}

\begin{enumerate}
\def\labelenumi{\arabic{enumi}.}
\item
  \textbf{Warm-up} oscillator ≥15 min (TCXO) or ≥30 min (watch crystal).
\item
  MCU captures rising-edge timestamps (e.g., 100 ms gate, 32-bit timer).
\item
  Stream timestamp, cycles CSV over USB; log with minicom or
  python-serial.
\item
  Simultaneously tap the RF line with logic analyzer; record 60 s bursts
  at 50 MS/s for FFT later.
\item
  After each run, save environment notes: box temp, supply voltage, room
  activity.
\end{enumerate}

\textbf{5 Data-Analysis Recipe (Python)}

python

CopyEdit

import pandas as pd, numpy as np, scipy.signal as ss, matplotlib.pyplot
as plt

df = pd.read\_csv('B1.csv', names={[}'t','N'{]})

f\_inst = df.N.diff()/df.t.diff() \# instantaneous freq

allan\_tau, allan\_dev, \_ = allantools.oadev(f\_inst, rate=10,
data\_type='freq')

\# FFT for χ side-bands

fs = 50e6

sig = np.load('burst\_B1.npy')

f, Pxx = ss.welch(sig, fs, nperseg=2**20, scaling='density')

side\_mask = (np.abs(f - f0/14) \textless{} 0.1) \textbar{} (np.abs(f +
f0/14) \textless{} 0.1)

peak = 10*np.log10(Pxx{[}side\_mask{]}.max())

\emph{Positive criterion}:

\begin{itemize}
\item
  Allan-dev bump at τ ≈ 10--120 s \textbf{and}
\item
  Side-band peak \textgreater{} --80 dBc at \emph{either} f0±f0/14f\_0 ±
  f\_0/14f0​±f0​/14
\end{itemize}

If both fail → TORUS-NEGATIVE for this construct.

\textbf{6 Expected TORUS Signal}

\begin{longtable}[]{@{}ll@{}}
\toprule
\textbf{Parameter} & \textbf{Nominal value}\tabularnewline
\midrule
\endhead
Carrier f0f\_0f0​ & 10 000 000 Hz (TCXO) / 32 768 Hz
(watch)\tabularnewline
χ-side-band offset & f0/14f\_0/14f0​/14 ≈ 714 285.7 Hz / 2340.57
Hz\tabularnewline
Predicted amplitude & --70 dBc \ldots{} --80 dBc
(persistent)\tabularnewline
Drift slope & f˙/f0∼1×10−6\textbackslash{}dot f/f\_0 \textbackslash{}sim
1 × 10\^{}\{-6\}f˙​/f0​∼1×10−6 over 1 min window\tabularnewline
\bottomrule
\end{longtable}

\emph{A null run should sit below --100 dBc and show white-FM Allan
slope.}

\textbf{7 Troubleshooting \& Noise Killers}

\begin{itemize}
\item
  Use shielded can + feed-through caps if mains hum shows in PSD.
\item
  Power from a linear bench supply (no laptop USB).
\item
  Place the MCU outside the Faraday box; bring coax through copper tape
  feed-through.
\item
  Compare two crystals in same box to cancel ambient temp drift
  (Δ-frequency method).
\end{itemize}

\textbf{8 Reporting Template}

\begin{longtable}[]{@{}ll@{}}
\toprule
\textbf{Field} & \textbf{Example}\tabularnewline
\midrule
\endhead
Crystal ID & Abracon ASTX-13-33-10.000 MHz\tabularnewline
Box Temp (°C) & 32.7 ±0.2\tabularnewline
Allan τ60 (baseline) & 2.1 × 10⁻¹⁰\tabularnewline
Allan τ60 (prepared) & \textbf{7.4 × 10⁻⁷}\tabularnewline
Side-band @ +f0/14 & --79.5 dBc (persistent 180 s)\tabularnewline
Verdict & TORUS-POSITIVE (β ≈ 125)\tabularnewline
\bottomrule
\end{longtable}

\textbf{9 Next If Positive}

\begin{itemize}
\item
  \textbf{Symbolic ladder residuals} -- feed your measured Δf into χ-β
  solver; cross-check with Catalan \& ζ(3) ratios.
\item
  \textbf{Halcyon sync test} -- stream live drift into a sandbox agent;
  watch for loss-cone collapse in learning curve.
\end{itemize}

\textbf{10 Next If Negative}

\begin{itemize}
\item
  Swap oscillator type (watch ↔ TCXO).
\item
  Run same protocol in a different lab or at a different latitude
  (geomagnetic sanity check).
\item
  Escalate to optical cavity (100 MHz) for extra decade of resolution.
\end{itemize}

\textbf{Ready for Bench Power-On}

Copy this sheet to the lab notebook, wire it up, and start logging.\\
Ping me with your first CSV or burst capture and I'll crunch the
Allan/FFT pipeline for you.

\end{document}
